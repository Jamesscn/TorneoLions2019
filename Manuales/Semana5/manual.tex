\documentclass{article}
\usepackage[utf8]{inputenc}
\usepackage[english]{babel}
\usepackage{graphicx}
\usepackage{float}
\usepackage{listings}
\usepackage{hyperref}
\usepackage{amsmath}
\hypersetup{
    colorlinks=true,
    linkcolor=blue,
    filecolor=magenta,      
    urlcolor=cyan,
}
\urlstyle{same}

\title{Manual 5 - 2do Torneo de Programación Competitiva}
\author{Lions R.C.}
\date{Agosto 2019}

\begin{document}

\maketitle

\tableofcontents

\begin{figure}[H]
    \centering
    \includegraphics[width=0.2\paperwidth]{newblack}
\end{figure}

\section{Programación oriendado a objetos}

\subsection{Structs}

\subsection{Union}

\subsection{Clases}

\section{Punteros}

\subsection{Referencias}

\section{Asignación de memoria}

\section{Estructuras y grafos con punteros}

\subsection{Listas encadenadas, dobles y cíclicas}

\subsection{Arboles de binario}

\section{Trucos de programación competitiva}

\subsection{Arboles de segmentos}

\subsection{Tablas hash}

\subsection{Números grandes}

\section{Enums}

\section{Manipulación de bits}

\section{Matemáticas modulares}

\subsection{Adición}

\subsection{Multiplicación}

\subsection{Exponenciación}

\end{document}